%% FRONTMATTER
\begin{frontmatter}

% generate title
\maketitle

\begin{abstract}

\noindent Music Information Retrieval (MIR) is an interdisciplinary science of extracting non trivial information and statistics from different sources of music. In today's digital age, music is stored in a variety of digitized formats - e.g midi, musicxml, mp3, digitized sheet music(in the form of images in .png and .jpeg formats) etc. Music Information Retrieval(MIR) systems aim at extracting features from one or more of these sources. MIR research helps in solving problems like automatic music classification, recommendation engine design etc. Users can then query the acquired statistics to obtain relevant information. \\\\
In this thesis, the author proposes and implements a new Music Information Retrieval and Structured Querying Engine called Modulo7. Unlike other MIR software which primarily deal with low level audio features, Modulo7 operates at a higher abstraction level, on the principles of music theory and a symbolic representation of music(by treating musical notes instead of acoustic pitches as the basic blocks of representation of musical data). Modulo7 is implemented as a full stack deployment, with server components that parse various sources of music data into its own efficient internal representation and a client component that allows consumers to query the system with SQL like queries which satisfies certain music theory criteria (and as a consequence Modulo7 has a custom relational algebra with its basic building blocks based on music theory) along with a traditional search model based on non trivial similarity metrics for symbolic music. Modulo7 also implements a lyrics analyzer, which supports functions such as lyrics similarity and meta dat prediction (e.g genre prediction)  \\\\ 


\vspace{1cm}

\noindent Primary Reader: Dr David Yarowsky\\

\end{abstract}

\begin{acknowledgment}

\noindent I would like to thank Dr David Yarowsky for giving me the opportunity to work on this project. His detailed insights have immensely helpful to me to power through my work and to also make the technical depth of the project accessible to laymen. I would like to thank Dr Yanif Ahmad for his crucial help in the systems aspects of my query engine and the implementation of the server side components. I would like to thank Natalie Draper in the Peabody Conservatory, my instructor for music theory for teaching me the basics of music theory. I would like to thank Dr Cory Mckay from McGill University for his help with understanding concepts in symbolic music Information Retrieval and help with implementation specifics for symbolic music information retrieval. I would like to thank Dr Ichiro Fuginaga from McGill University for his guidance and help with Optical Music Recognition concepts. \\\\
I would like to thank my friends Aakash, Ankit, Satya and Japneeth for their support, encouragement and help with technical and coding aspects of the project. \\\\
Most importantly I would like to thanks my family for their unending support and faith in me, and for instilling a love for music which has allowed me to take this in depth study of applications of Computer Science to Music theory.

\end{acknowledgment}

\begin{dedication}
 
This thesis is dedicated to my family and to all the music lovers in the world. 

\end{dedication}

% generate table of contents
\tableofcontents

% generate list of tables
\listoftables

% generate list of figures
\listoffigures

\end{frontmatter}
