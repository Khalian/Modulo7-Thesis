%% FRONTMATTER
\begin{frontmatter}

% generate title
\maketitle

\begin{abstract}

Music Information Retrieval (MIR) is an interdisciplinary science of extracting non trivial information and statistics from music data sources. In today's digital age, music is stored in a variety of digitized formats - e.g midi, musicxml, mp3, digitized sheet music etc. Music Information Retrieval Software aim at extracting features from one or more of these source. MIR research helps in solving problems like automatic music classification, recommendation engine design etc. Users can then query the acquired statistics to acquire relevant information. \\\\
The author proposes and implements a new Music Information Retrieval and Query Engine called Modulo7. Unlike other MIR software which deal with low level audio features, Modulo7 operates on the principles of music theory and a symbolic representation of music. Modulo7 is a full stack deployment, with server components that parse various sources of music data into its own efficient internal representation and a client component that allows consumers to query the system with sql like queries which satisfies certain music theory criteria (and as a consequence Modulo7 has a custom relational algebra with its basic building blocks based on music theory). 

\vspace{1cm}

\noindent Primary Reader: Dr David Yarowsky\\
Secondary Reader: Dr Yanif Ahmad

\end{abstract}

\begin{acknowledgment}

I would like to thank Dr David Yarowsky for giving me the opportunity to work on this project. His detailed insights have immensely helpful to me to power through my work. I would like to thank Dr Yanif Ahmad for his crucial help in the systems aspects of my query engine and the implementation of the server side components. 

\end{acknowledgment}

\begin{dedication}
 
This thesis is dedicated to my family and to all the music lovers in the world. 

\end{dedication}

% generate table of contents
\tableofcontents

\end{frontmatter}
