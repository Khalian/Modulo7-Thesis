\chapter{Basics of Music Theory}
\label{sec:intro}

\noindent Music theory is defined as the systematic study of the structure, complexity and possibilties of what can be expressed musically. More formally its the academic discipline of studying the basic building blocks of music and the interplay of these blocks to produce complex scores (pieces of music). Modulo7 is built on top of western theoretic principles and hence only western music theory is explored. Also music theory is an extremely complicated subject and hence only the basics and relevant portions to the modulo7 implementation are discussed here. \\\\
Traditionally music theory is used for providing directives to a performer to play a particular song/score. \\\\
This section is primarily meant for people with a weak or lack of understanding of music theory. 
The following section talks about the basic building blocks of music theory:-
\section{Building Blocks}

\noindent Music is built on fundamental quantities (much like matter is built on fundamental quantities like atoms and molecules). The following are the core concepts in order of atomicity (i.e successive blocks build on the preceding ones)

\subsection{Pitch/Note}

\noindent A pitch is a deterministic frequency of sound played by a musical voice (instrument or human). In western music theory, certain deterministic pitches are encoded as Notes. For example the note A4 is equal to 440 Hz. In other words Notes are symbolic representations of certain pitches. With certain notable exceptions, most music is played on these set frequencies. \\\\
Each note is characterized by two entities. First is the note type and the second is the octave. An octave can be considered as a range of 12 notes. There are 8 octaves numbered 0 to 7 which are played by traditional instruments or vocal ranges. Then is the note type. Notes are categorized into 7 major notes (called A, B, C, D, E, F, G) and 5 minor notes (also called as accidentals). They can be characterized by increasing or decreasing the frequency of the notes by a certain amount (called sharps(\#) and flats(b) respectively). For example the accidental lying in between (A and B is called A\# or Bb). Similarly accidentals lie in between C, D; D, E; F, G and G, A. (Note that there are no accidentals in between B and C and E and F).

\subsection{Semitone and Tone}
\noindent A semitone is an increment or a decrement between two notes. For instance there is one semitone in between A and A\#. Similarly there are 3 semitones in between A and C. A tone is an increment in between two major notes. Another characterization of a tone is two semitones.

\subsection*{Chord}
\noindent A chord is a set of notes being stacked together (being played together at or almost at the same time). Chords are the basic building blocks of a concept called harmony (which will be discussed further on.). Traditionally a chord is constructed by stacking together notes played on a single instrument, but a chord can be constructed by different instruments simultaneously playing different notes. 

\subsection{Rests}
\noindent Rests are pauses in between notes (with no sound being played at that point of time) for a fixed duration.

\subsection{Melody}
\noindent A melody is a succession of notes and rests which sound pleasing. There are many rules about what makes a melody sound good which we will get to in the subsequent reading.

\subsection{Harmony}
\noindent A harmony is a succession of chords (also known as a chord progression). 

\subsection{Voice}
\noindent A voice is an interplay of notes, chords and stops by a single instrument/vocalist. The reader can think of a voice as a hybrid or generalization of the melody and harmony concepts. 

\subsection{Score/Song}
\noindent A score or a song is an interplay of voices. It is the final product of music that is delivered to the audience. Songs are of different types based on cultural context and complexity (for example an orchestra is a large number of voices being coordinated by a conductor. In contrast a folk song might be played by a single person on a guitar or a duet between a vocalist and an instrumentalist). 
