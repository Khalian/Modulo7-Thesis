\chapter{Conclusions}

\noindent In this thesis, a new Music Information Retrieval system is proposed and implemented which applies concepts of music theory for structured querying and search based on custom similarity measures. \\

\noindent The goals that Modulo7 was able to accomplish can be stated as follows
\begin{enumerate}
\item To implement an space efficient and an universal indexing scheme for variegated sources of music.
\item To implement and expose a querying language and a search engine  which uses music theoretic criteria as its building blocks. 
\item To implement a lyrics analyzer to support lyrics similarity. To evaluate the lyrics similarity engine and its ability to predict meta data (in particular genre of a song).  
\item To explore and quantify the efficiency and efficacy of applying music theoretic concepts for similarity judgments and querying.
\end{enumerate} 

\noindent In a nutshell, Modulo7 unifies disparate music sources into one cohesive framework which allows for a common ground for querying and similarity searches on a heterogeneous data set.