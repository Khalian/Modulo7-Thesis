\chapter{Literature Review}

\noindent Music Information Retrieval is an active and vibrant community. Both academia and industry diligently pursue it albeit with different goals in mind. While academia's primary aim is to explore particular problems (e.g cover song detection, estimating chords from chroma vectors \cite{chord-detection} ) etc. Whereas Industry is primarily interested in solving problems like song recommendation and similarity searches. For this purpose, both communities have created an extensive set of software for solving these problems. The author presents an overview of such software and the problems they attempt to address. \\\\

\subsection{jMIR}

\noindent jMIR, or Java Music Information Retrieval tool set \cite{jMIR} is a collection of java code, GUI, API and CLI tools for the purpose of feature extraction from variety of music sources (in particular audio, midi) and mine cultural information from the web.