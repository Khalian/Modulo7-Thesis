\chapter{Introduction}
\label{sec:intro}

\noindent Why does a person like a particular song? What are the inherent aspects of a song that pleases a person's musical taste? Is it the complexity of a song, the beat of the song or just a particular melodic pattern that they find catchy? More so if a person likes a song, can we predict if he/she will like a similar song? If yes, then how is this similarity judged? \\\\
Music has been created since the dawn of civilization and these questions have plagued mankind just as long. In response to this, man has created elaborate systems of formal study for music and classification techniques in almost every ethnic community since antiquity. Two notable examples are the western system of solfege and classical music theory and the Indian system of raagas. These elaborate systems are based on very simple fundamental building blocks of melody and harmony and simple rules that govern the interplay of these building blocks. However very complex pieces of music can be created with these simple rules depending on the skill and virtuosity of artists. Likewise, composers use these rules and concepts to create novel music for mass consumption. \\\\
In the modern era industry and academia have attempted to address the problem of music recommendation and music classification. Industry has predominantly favored approaches that look at user preferences and history as a basis of prediction and recommendation. For example Amazon Music recommendation works on consumer behavior (user's shopping, browsing history and related consumer behavior \cite{amazonreco}). Pandora on the other hand utilizes musicologists to ascertain how a song is similar to another song and creates software that leverages this ad-hoc generated graph of similarity \cite{musicgenomepandora}. These approaches are either expensive in the human labor needed or in the amount of data processed that is input from a large number of users. More recently, companies like Echo Nest have extensively extracted features from music sources \cite{echonestfingerprint} and mined cultural information on the web but leave it on the consumers to determine how best to leverage this extracted data. Hence symbolic MIR is not traditionally used in industry and music theory is an after thought in almost all industry applications. \\\\
Academia on the other hand attempts to solve very particular problems in MIR. Typical examples would be cover song detection\cite{coversongid}, processing information via signal processing, audio feature extraction, optical music recognition \cite{omrsurvey} etc. In most cases the applications are of a very specific domain and does not fully scale with bulk music data. Generic frameworks like the jMIR \cite{jMIR} (which also happens to be a major inspiration for Modulo7) suite for automatic music classification exists, which is meant to facilitate research in MIR with a machine learning focus. However academia is disconnected with industry and no full scale MIR engines exists in academia which can satisfy the scale of industry applications. \\\\
This work is attempt to bridge both communities. Modulo7 is a full stack deployment of Music Information Retrieval Software, providing both a server architecture and a SQL like client and a search engine functionality to find relevant songs on music theoretic criteria. Modulo7 does not attempt to solve very complex music theoretic problems (e.g study orchestral music to identify counter point class). Rather Modulo7 acts a framework on which such analysis can be built upon. Most importantly, Modulo7 addresses the issue of scale and allows a fast and efficient comparison between songs. It also addresses deficiencies in existing software, such as predicting incomplete meta data information in music sources. Particular examples for this would be Key estimation, Tempo estimation etc. \\\\
Modulo7 implements a unique indexing scheme and a universal "document" representation of music. This indexing scheme involves creating an inverted index for global properties of songs (key signature, the property of homophony, time signature etc). This indexing scheme allows for fast lookups for certain types of queries (e.g find all songs that in the key of C Major) and also allows for speedup in scenarios which require criteria based on indexed terms. 