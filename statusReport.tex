\documentclass[11pt]{article}
\usepackage{latexsym}
\usepackage{amssymb}
\usepackage{times}
%\usepackage[in]{fullpage}
\usepackage{amsmath,amsfonts,amsthm}
\usepackage{array}
\usepackage{multirow}
\usepackage{tikz}
\usepackage{enumitem}
\usepackage{graphicx}


\oddsidemargin 0mm
\evensidemargin 5mm
\topmargin -20mm
\textheight 240mm
\textwidth 160mm

%\documentclass[11pt]{article}
%\pagestyle{myheadings}
%\usepackage[ruled,nothing]{algorithm}
%\usepackage{algorithmic}
%\usepackage[dvips]{epsfig,graphicx}
%\numberwithin{equation}{section}

\newenvironment{newalgo}[2]{\begin{algorithm}

\caption{\textsc{#1}}\label{#2}

\begin{algorithmic}[1]}{\end{algorithmic}\end{algorithm}}

\newcommand\MyBox[2]{
  \fbox{\lower0.75cm
    \vbox to 1.7cm{\vfil
      \hbox to 1.7cm{\hfil\parbox{1.4cm}{#1\\#2}\hfil}
      \vfil}%
  }%
}

\newcommand{\gm}{\gamma}
\newcommand{\wh}{\widehat}
\newcommand{\rep}{representation}
\newcommand{\rv}{random variable}
\newcommand{\la}{\lambda}
\newcommand{\wt}{\widetilde}
\newcommand{\st}{such that}
\newcommand{\slvary}{slowly varying}
\newcommand{\ma}{moving average}
\newcommand{\regvary}{regularly varying}
\newcommand{\asy}{asymptotic}
\newcommand{\ts}{time series}
\newcommand{\id}{infinitely divisible}
\newcommand{\seq}{sequence}
\newcommand{\fidi}{finite dimensional \ds}

\newcommand{\ble}{\begin{lemma}}
\newcommand{\ele}{\end{lemma}}
\newcommand{\bfX}{{\bf X}}
\newcommand{\pro}{probabilit}
\newcommand{\BX}{{\bf X}}
\newcommand{\BY}{{\bf Y}}
\newcommand{\BZ}{{\bf Z}}
\newcommand{\BV}{{\bf V}}
\newcommand{\BW}{{\bf W}}
\newcommand{\reals}{{\mathbb R}}
\newcommand{\bbr}{\reals}

\newcommand{\balpha}{\mbox{\boldmath$\alpha$}}
\newcommand{\bbeta}{\mbox{\boldmath$\beta$}}
\newcommand{\bmu}{\mbox{\boldmath$\mu$}}
\newcommand{\tbmu}{\mbox{\boldmath${\tilde \mu}$}}
\newcommand{\bEta}{\mbox{\boldmath$\eta$}}


\def \br#1{\left \{#1 \right \}}
\def \pr#1{\left (#1 \right)}

\newcommand{\Gm}{\Gamma}
\newcommand{\ep}{\epsilon}


\newtheorem{lemma}{Lemma}[section]
\newtheorem{figur}[lemma]{Figure}
\newtheorem{theorem}[lemma]{Theorem}
\newtheorem{proposition}[lemma]{Proposition}
\newtheorem{definition}[lemma]{Definition}
\newtheorem{corollary}[lemma]{Corollary}
\newtheorem{example}[lemma]{Example}
\newtheorem{exercise}[lemma]{Exercise}
\newtheorem{remark}[lemma]{Remark}
\newtheorem{fig}[lemma]{Figure}
\newtheorem{tab}[lemma]{Table}
\newtheorem{fact}[lemma]{Fact}
\newtheorem{test}{Lemma}
\newtheorem{algorithm}[lemma]{Algorithm}

\newcommand{\play}{\displaystyle}

\newcommand{\ms}{measure}
\newcommand{\beao}{\begin{eqnarray*}}
\newcommand{\eeao}{\end{eqnarray*}\noindent}
\newcommand{\beam}{\begin{eqnarray}}
\newcommand{\eeam}{\end{eqnarray}\noindent}

\newcommand{\halmos}{\hfill\mbox{\qed}\\}
\newcommand{\fct}{function}
\newcommand{\ins}{insurance}

\newcommand{\one}{{\bf 1}}
\newcommand{\eid}{\buildrel{\rm d}\over {=}}
\newcommand {\Or}{\rm ORDER}
\newcommand {\In}{\rm INTER}

\newcommand{\bbd}{{\mathbb D}}
\newcommand{\vi}{$V_{ij}$ }
\newcommand{\rr}{R^{\prime\prime}}
%\newcommand{\R}{R^\prime}
\newcommand{\ci}{\frac{1}{c}}
\newcommand{\Vi}{V(n)}
\newcommand{\dR}{\mathcal R}
\newcommand{\md}[1]{\left(\ \rm{mod}\ \it{#1}\right)}
\newcommand{\So}{s}
\usepackage{algpseudocode, caption}
\usepackage[ruled]{algorithm}
\usepackage{hyperref}
%\begin{document}
%\def\DoubleSpace{\baselineskip=24pt}
%\DoubleSpace \sloppy

\begin{document}


\title{Arunav Sanyal Status Report on thesis project : Modulo7 }

\maketitle

%%%%%%%%%%%%%%%%%%%%%%%%%%%%%%%%%%

\begin{enumerate}

\item Completed the playback module as requested (plays back songs present in the similar songs returned). \textbf{There is a depreciable lack of quality for some formats e.g. musicxml and midi, would that be appropriate or should I write a crawler which finds an equivalent mp3 and plays it?}. 

\item I successfully ran my first large scale experiment. The experiment involved estimating key signatures for 1314 music xml songs via an algorithm implemented in Modulo7 and reported accuracy of 85.9 where the baseline is 50 percent. 

\item Implementing reporting similarity scores while running the similarity metric mode for Modulo7. \textbf{Smith Waterman Algorithm reports alignment scores in large integer numbers instead of the similarity defined between 0 and 1, I will fix that}

\item Implemented a couple of n gram similarity measures SCM Trigam similarity and ukkonnen measure successfully

\item Experiments for melodic similarity (the ultimate goal for this project) are in a bit of a pickle. I still am not able to get hand of the data present in literature due to non responsive authors. \text{Perhaps you can drop them a mail}. I reached out to peabody faculty to help me out in this matter as well.

\end{enumerate}

\end{document}

%%%%%%%%%%%%%%%%%%%%%%%%%%%%%
