\chapter{Mathematics of Modulo7}
\label{sec:mir math}

\noindent The following sections describe the mathematical concepts used and implemented in Modulo7.

\section{Vector Space Models of Music}

\noindent In traditional text based information retrieval retrieval systems, documents are indexed and a vector space representation of documents are created. Typical approaches for counting term frequencies or some weighting scheme like Term Frequency-Inverse Document Frequency Approach (TF-IDF). Analogous to text based IR, Music data can also be expressed as a vector space based on the approach taken. Some of these approaches are taken from the SIMILIE \cite{similietechnicalmanual} but generalized for polyphonic music. Many approaches are novel based on the author's music theoretic studies. 

\subsection{Vector Space Models per voice}

Certain vector space models are with respect to a single voice. This allows vector space models to be 

\subsection{Pitch Vector}

\noindent A voice can be expressed as a sequence of pitches $n_i$ = ($p_i, t_i$) where $p_i$ is the pitch and/or the set of pitches at instant of time $t_i$. The symbolic representation of music essentially discretizes these values from music sources and hence a vector representation can be made. A voice V can be represented as a vector :-

\begin{equation}
P = <n_1, n_2, ... n_n>
\end{equation}

\noindent A similar vector representation could be when the time information is eschewed in favor on only the pitches. This vector is called the raw pitch vector and is denoted as the follows :-

\begin{equation}
R = <p_1, p_2, ..., p_n>
\end{equation}

\subsection{Pitch Interval Vector}

\noindent Another way to look at elements is the interval spacing between elements. This is same as the interval concept in the music theory chapter. Mathematically an interval is defined as $\Delta p_i = p_i - p_{i-1}$. And thus an pitch interval vector is defined as :-

\begin{equation}
PI = <\Delta p_1, \Delta p_2, ... , \Delta p_n>
\end{equation}

\subsection{Rhythmically weighted Pitch Interval Vector}

\noindent In order to include the rhythmic information in the pitch interval Vector, define rhythmically weighted pitch as $rp_i = \Delta p_i \times t_i$. Now the rhythmically weighted pitch vector can be represented as:-

\begin{equation}
RPI = <rp_1, rp_2, ... rp_n>
\end{equation}

\subsection{Normalized Tonal Histogram Vector}

\noindent The tonal histogram is a vector or map of 12 distinct intervals present in western music theory. Each position in the vector corresponds to the total number of times that interval has occurred in a voice. Mathematically define $\Delta P_i^{voice_j} = \sum_{i=1}^{len(voice)} p_i^{voice_j}$. Moreover 
