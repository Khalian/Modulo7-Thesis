\chapter{Mathematics of Music Information Retrieval}
\label{sec:mir math}

\noindent The following sections describe the mathematical concepts used and implemented in Modulo7. 

\section{Vector Space Models of Music}

\noindent In traditional text based information retrieval retrieval systems, documents are indexed and a vector space representation of documents are created. Typical approaches for counting term frequencies or some weighting scheme like Term Frequency-Inverse Document Frequency Approach (TF-IDF). Analogous to text based IR, Music data can also be expressed as a vector space based on the approach taken. Some of these approaches are taken from the SIMILIE \cite{similie} but generalized for polyphonic music. Many approaches are novel based on the author's music theoretic studies. 

\subsection{Pitch Vector}

\noindent A voice can be expressed as a sequence of pitches ($p_i, t_i$) where $p_i$ is the pitch at instant of time $t_i$. The symbolic representation of music essentially discretizes these values from music sources and hence a vector representation can be made:-



